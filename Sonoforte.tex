\documentclass{article}
\usepackage{graphicx}
\usepackage{amsmath}
\usepackage{cite}
\usepackage{color}
\usepackage{enumitem}
\usepackage{natbib}
\usepackage{eurosym}
\usepackage{hyperref}
\newcommand{\mail}[1]{\href{mailto:#1}{\texttt{#1}}}
\newcommand{\equat}[2]{\href{https://it.wikipedia.org/}{\texttt{#2}}}

\title{Prova esercizio}
\author{Alessandro Meloni}
\date{29/10/2023}

\begin{document}
\maketitle
\begin{abstract}
    \textit{Questo articolo parlerà di come io sia riuscito a superare le mie ansie e paure} , grazie all'ausilio di una persona a me fidata.\citep{Brooks1997Methodology}
    \textbf{Tutto ciò deve essere d'aiuto alle persone} che, come me, hanno sofferto di questo disturbo per \textsc{svariati motivi} , o, che hanno vissuto \texttt{in famiglia delle situazioni} che hanno cambiato il loro di modo di vivere e di approccio alla vita.
\end{abstract}
\tableofcontents
\newpage
  \centering
  \section{La mia infanzia}
  \flushleft \emph{Ciao!} , Io mi chiama Alessandro.\citep{Karthik2001Analysis}

\newpage
   \centering
   \section{La mia adolescenza}
   \flushleft Sono sempre stato un ragazzo per bene, preso d'esempio un po' da tutti i miei amici.\href{https://github.com/}{Entra su GIT}
   
\newpage
\begin{equation*}
   a = b^{5_2} * (\mu * \Omega)
\end{equation*}

\begin{center}
    \begin{math}
    \frac{\operatorname{C*v*5}5^2}
        {\operatorname{\alpha*\beta}}
\end{math}
\end{center}

\begin{align}
    a*b & = 5 * 10\\
    hfh & = 10 * 10
\end{align}
\begin{center}
  \begin{tabular}{|c|c|}\hline 
   SQUADRE & Punti \\ \hline
   \includegraphics[width = 0.1\linewidth]{Images/rubentus.jpeg} Rubentus & 20 \\ \hline
   \includegraphics[width = 0.1\linewidth]{Images/fozzainda.jpg} Inda & 17 \\ \hline
   \includegraphics[width = 0.1\linewidth]{Images/fozzamila.jpeg} Mila & 16 \\ \hline
\end{tabular}\\  
\end{center}

La tabella presenta le seguenti scelte:
\begin{enumerate}
\item Rubentus vince
\item Inda vince
\item Mila vince
\end{enumerate}
\begin{tabular}{|c|c|c|}\hline

    Email & url & equazione \\ \hline
    \mail{ales.sgrogna@peto.it} & \url{https://github.com/} & \equat{https://it.wikipedia.org/}{Mandami al link} \\ \hline
    
\end{tabular}
\centering
\bibliography{Database}
\bibliographystyle{plainnat}
\end{document}