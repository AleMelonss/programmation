\documentclass{article}
\usepackage{graphicx}
\usepackage{amsmath}
\usepackage{cite}
\usepackage{color}
\usepackage{enumitem}
\usepackage{hyperref}
\usepackage{natbib}
\usepackage{tabularx}
\usepackage{natbib}
\usepackage{ragged2e}

\title{ANAC: Anticorruzione e Digitalizzazione}
\author{Alessandro Meloni GEPID}
\date{Anno accademico 2023/2024}

\begin{document}
\maketitle

\centering \tableofcontents
\newpage\centering
\section{Introduzione}
\flushleft 
\begin{justify}
Ho deciso di presentare questo argomento, che comunque risulta essere molto ampio, soprattutto con lo sviluppo delle tecnologie odierne, e, la quale può aprire un forte dibattito. Specificatamente l'Anac è un ente alla quale mi sono avvicinato durante il mio periodo di studio in triennale, avendo fatto il tirocinio presso l'Ufficio Anticorruzione del Comune di Cagliari. È un ente molto complesso e che richiede un grosso impegno nel poterlo comprendere, e che sicuramente è uno dei settori che potrebbe essere maggiormente colpito (sia in positivo che in negativo, dipende dalla propria opinione) dall'avvento di tecnologie quali Blockchain, Ai, rispetto ai normali database che vengono utilizzati attualmente. Con questo paper cerco di dare una mia opinione e porre un'analisi oggettiva su quali potrebbero essere le ripercussioni, tenendo conto dell'introduzione del nuovo codice degli appalti.
\end{justify}
\newpage\centering
\section{ANAC}
\flushleft Esempio.

\flushleft \subsection{Storia dell'ente}
Esempio.
\flushleft \subsection{Funzioni dell'ente}
Esempio.
\newpage\centering
\section{Codice degli appalti}
\flushleft Esempio.

\flushleft \subsection{Nuovo codice degli appalti}
Esempio.
\flushleft \subsection{Digitalizzazione dei contratti}
Esempio.
\flushleft \subsection{Rapporto blockchain e contratti pubblici}
Esempio.

\newpage \centering
\section{Conclusione e opinioni finali}
\flushleft Esempio.

\end{document}
