\documentclass{article}
\usepackage{graphicx}
\usepackage{amsmath}
\usepackage{cite}
\usepackage{color}
\usepackage{enumitem}
\usepackage{natbib}
\usepackage{tabularx}
\usepackage{natbib}
\usepackage{ragged2e}
\usepackage{geometry}
\usepackage{url}
\geometry{a4paper, left=3.5 cm, right=3.5 cm, top=3 cm, bottom=3 cm}

\title{\huge\textbf{La digitalizzazione nelle PA}}
\author{\texttt{Alessandro Meloni 1118676 GEPID}}
\date{Anno accademico 2023/2024}

\renewcommand\contentsname{Indice}
\renewcommand\refname{Bibliografia}

\begin{document}
\begin{figure}
    \centering
    \includegraphics[width=0.9\linewidth]{Uniboarms.png}
\end{figure}
\maketitle

\centering \tableofcontents

\newpage\centering
\section{Introduzione}
\begin{justify}
L'ampliamento di tecnologie che usufruiscono di intelligenza artificiale è un fatto che agli occhi di tutti è percepito in questo preciso momento storico. L'errore che si fa però è pensare che l'AI sia un fatto recente, in realtà i primi studi pubblicati sulle reti neurali risalgono al 1943, pertanto gli studi hanno circa 50 anni di storia.\citep{mcculloch1943logical}\\ Usufruiscono di sistemi di intelligenza artificale anche il riconoscimento delle spam nella posta elettronica (anni 90); oppure ancora il riconoscimento della caligrafia. Semplicemente ora siamo in un momento storico dove si è compresa l'enorme potenzialità di questi algoritmi e la loro capacità di poter ‘‘ingerire’’ informazioni, riuscendole a processare in modo ottimale.\\
L'Italia in termini di investimenti risulta essere un po' indietro rispetto ad altre realtà europee, però parallelamente si sta facendo un grosso passo avanti nel cercare di far coincidere l'algoretica con i bisogni economici digitali: per \textit{algoretica} si intende riuscire a far combaciare l'utilizzo degli algoritmi e dell'etica in modo tale che non si vadano a creare delle soluzioni disumanizzate (in sostanza abbiamo tanta ricerca ma poca innovazione).
Di fatto lo sviluppo delle tecnologie odierne, sappiamo tutti, sta aprendo un forte dibattito, pertanto, risulta opportuno avere una corretta regolamentazione sia a livello statale che sovrastatale.
In Italia ci sono varie aurorità che si occupano del settore della digitalizzazione, ognuno con compiti specifici.
Sono tutti enti molto complessi che tentano di accompagnare l'Italia verso una completa transizione digitale.\\
Nello specifico si analizzeranno gli articoli che sanciscono i principi digitali del nuovo codice dei contratti pubblici (d.lgs 36/2023), prevalentemente nell'art 30.\\
L'articolo parla dell'utilizzo di meccanismi automatizzati nel ciclo di vita dei contratti pubblici, ragion per cui bisognerà preventivamente comprendere i principi enunciati dagli artt 19 al 36.\\
Trasversalmente verranno anche spiegate le funzioni dell'ANAC e la regolamentazione delle Autorità Amministrative Indipendenti, ma senza allontanarsi troppo dal topic centrale, così da dare una visione di insieme più corretta e comprensibile.\\
Infine si vedrà a che punto è la regolamentazione UE sull'AI e gli obiettivi che si sono raggiunti in Italia attraverso l'ausilio dei fondi del PNRR.
Con questo paper si cerca di analizzare oggettivamente quali potrebbero essere le ripercussioni a livello organizzativo e nel mondo del lavoro, pur sempre nel rispetto dei principi amministrativi.
\end{justify}

\centering
\section{Nuovo codice degli appalti}
\begin{justify}
    Il d.lgs 36/2023 è il continuum del d.lgs 50/2016 (il suo predecessore in materia di contratti pubblici), il quale ambito viene gestito direttamente dall'ANAC: autorità amministrativa indipendente che si occupa della prevenzione alla corruzione, rispetto degli obblighi di trasparenza delle PA, affidamento ed esecuzione dei contratti pubblici in tutto il territorio italiano.\citep{AnacSite}\\
    Al contrario delle precedenti versioni, in questo c'è stato un maggiore coinvolgimento di figure professionali cardine nel garantire una corretta transizione digitale, figure come ingegneri, informatici, che sicuramente avranno aiutato a comprendere al meglio tutti gli aspetti tecnici (visto che uno dei punti focali è la digitalizzazione dell'intero ciclo di vita dei contratti).\\
    Chiaramente l'attività è avvenuta anche in coordinamento delle discipline già presenti in materia di digitalizzazione, come il codice dell'amministrazione digitale (d.lgs 82/2005).
\end{justify}

\newpage\subsection{Digitalizzazione dei contratti e principi}
\begin{justify}
    Come accennato in precedenza, la parte centrale della riforma del codice è proprio la completa digitalizzazione delle fasi di vita dei contratti, sicchè, una serie di principi sono stati stabiliti a descrizione del processo.
    Tale codice ha efficacia operativa dal 1° Gennaio 2024, obbligando tutte le PA e soggetti interessati a conformarsi.\\
    \begin{enumerate}
        \item \texttt{Interoperabilità:} principio che ha una grossa rilevanza sopratutto nel rispetto dell'unicità dell'invio dei dati, ciò perchè consolida che il privato o imprese non abbiano l'obbligo di presentare un ulteriore volta dati che sono stati già inviati in precedenza a una PA, questo incentiva le amministrazioni ad avere una maggiore collaborazione. Le piattaforme dovranno essere interoparabili a partire dal 1° Gennaio 2025, nei confronti di tutte le stazioni appaltanti con interventi al di sopra di $1.000.000$ di euro(\textit{art 19}).
        \item \texttt{E-procurement:} correlato al principio enunciato poc'anzi, sottolinea quanto i dati e informazioni debbano essere prese attraverso una serie di piattaforme e servizi digitali di cui usufruisce l'amministrazione, legati o meno alla vita dei contratti pubblici.\\ Tra le piattaforme ci sono le banche dati degli enti che danno la possibilità di reperire i dati attraverso altre amministrazioni, rispettando il principio di unicità.\\
        ANAC è titolare della \textit{Banca dati nazionale dei contratti pubblici} la quale determina quali sono le informazioni che devono passare obbligatoriamente per essa, e, per eventuali omissioni informare l'AGID che agirà di conseguenza tramite i suoi poteri sanzionatori. I dati sono accompagnati da dei fascicoli virtuali dell'operatore economico, all'interno della quale sono inseriti tutti i dati e le procedure di gara a cui ha partecipato tale operatore. Infine, se le stazioni appaltanti non hanno delle piattaforme nominative possono usufruire di quelle di altre stazioni; oltretutto, non si può impedire l'accesso ai dati se il comportamento dell'operatore coinvolto distorce la concorrenza (con l'ultimo aggiornamento delle linee guida sull'interoperabilità AGID, prevede anche la possibilità di informare AGID, definendo l'esigenza e formulando eventuali proposte che potranno poi attivare una consultazione tra le parti\citep{AGID_linee_2023}). Come stabilito dal codice:‘‘\textit{avviene un'acquisizione diretta dei dati e delle informazioni inseriti nelle piattaforme, ai sensi degli articoli 3-bis e 22 e seguenti della legge 241/1990 e degli articoli 5 e 5-bis del decreto legislativo 33/2013.... ad esclusione dei contratti che richiedono speciali misure di sicurezza}(contratti secretati)’’.\\
        Di fatto, tutte le comunicazioni devono avvenire seguendo la logica delle piattaforme e/o a mezzo di domicilio digitale (\textit{artt 22,23,24,25,29}).
        \item \texttt{Trasparenza:} il codice rimanda al d.lgs 33/2013 in materia di pubblicità e trasparenza.
        Attraverso la pubblicazione nelle sezioni di amministrazione trasparente di ogni ente con collegamento verso la banca dati ANAC, viene rispettato il principio di pubblicità legale, inoltre, si effettua anche una tempestiva comunicazione presso l'ufficio di pubblicazione dell'UE.
        Questo evidenzia quanto sia rilevante garantire la tracciabilità, la possibilità effettiva di essere a conoscenza delle attività e dei dati interconnessi. (\textit{artt 19, 20, 27, 28}).
        \item \texttt{Protezione dei dati personali e sicurezza informatica:}
        Le regole tecniche sono stabilite di comune accordo tra AGID, ANAC e Presidenza del consiglio, devono essere garantite tutte le misure necessarie a presidio della sicurezza delle infrastrutture di rete utilizzate e dei dati personali ivi presenti. Tutto ciò avviene a seguito di un costante aggiornamento del personale, facendo capo ad algoritmi crittografici ed opportune valutazioni di rischio (con personale specializzato per ogni ente/stazione coinvolta) \textit{(artt 19, 25, 26)}. \\Deduciamo quindi che più aumenta il grado di digitalizzazione, maggiore sarà la possibilità di subire attacchi cyber.\\
        Secondo il report annuale redatto dai servizi segreti italiani, è emerso che circa il 43\% degli attacchi cyber ha come destinatario obiettivi pubblici, precisamente viene fatto leva sulla vulnerabilità degli accessi di soggetti che usufruiscono di servizi della PA emessi da società esterne.\citep{ChiomentiPA}\\
    \end{enumerate}
    Con l'obiettivo di semplificare il rapporto tra cittadini, imprese e PA tramite tecnologie digitali, l'unione di questi principi da vita al \textit{diritto di cittadinanza digitale}, che già in parte viene garantito attraverso l'ausilio del codice dell'amministrazione digitale (d.lgs 82/2005).
\end{justify}

\subsection{Art 30 d.lgs 36/2023}
\begin{justify}
    Si è voluto dedicare un paragrafo a parte sull'art 30 perchè è l'articolo relativo alle decisioni automatizzate, perciò collegato a l'utilizzo di meccanismi di intelligenza artificiale.\\
    Da tale articolo è desumibile il principio della \textit{neutralità tecnologica}, che si riferisce alla dominanza dell'uomo sulle tecnologie, il quale tramite il suo controllo, deve riuscire ad utilizzarle in funzione delle necessità dell'operatore/amministrazione, in modo tale da progredire in termini di efficacia ed efficienza dei processi, senza mai rendere più difficoltoso o imparziale tale funzionamento.
    Alla base di questa definizione è opportuno precisare quali potrebbero essere i sottoprincipi della neutralità e le caratteristiche che l'algoritmo utilizzato può assumere.\\
    I sottoprincipi deducibili dalla neutralità sono i seguenti:
    \begin{itemize}
        \item \texttt{Non discriminazione algoritmica}: qualsiasi attività venga compiuta con l'ausilio di queste tecnologie, non deve mai essere discriminante in nessun ambito (genere, religioso, lavorativo etc..).
        \item \texttt{Non esclusività algoritmica}: nessuna attività può essere compiuta senza il sostegno della figura umana, che deve essere obbligatoriamente presente in ogni settore, per cui non possono essere lasciati ambiti decisionali a un'intelligenza artificiale. L'unico caso possibile è nelle risposte automatizzate di sistema, ma anche in questo caso, secondo \textit{l'art 18 del ‘‘Codice europeo di buona condotta amministrativa’’}, nel caso venga richiesto espressamente dai destinatari, devono opportunamente essere elaborate delle risposte specifiche da parte del personale preposto a ciò.
        \item \texttt{Conoscibilità dell'algoritmo}: qualsiasi strumento informatico utilizzato per coadiuvare le attività svolte da un operatore/amministrazione deve essere obbligatoriamente reso pubblico, così da essere il più esaustivi possibili nelle spiegazioni e nella comprensione delle tecnologie utilizzate in materia. Nel caso non si possa rendere pubblico il codice sorgente, deve essere pubblicata qualsivoglia documentazione relativa all'algoritmo, con opportuno collegamento presso la sezione di \textit{Amministrazione trasparente}.
    \end{itemize}
    Oltre ai sottoprincipi è utile farsi un'idea anche dei tipi di algoritmi:
    \begin{itemize}
        \item \texttt{Algoritmi deterministici}: sono quel tipo di algoritmi che, secondo un meccanismo di causalità, basano le loro scelte su una concatenazione di atti prestabiliti l'uno seguente a l'altro.
        \item \texttt{Algoritmi non deterministici}: computazioni che presentano uno o più passi\\ dell'algoritmo, possedendo ciascuno di essi due o più alternative di successione applicate secondo una logica probabilistica.
        \item \texttt{Algoritmi ad apprendimento automatico}: sono delle tecniche di AI, che, sulla base di una serie di dati, apprende in funzione di una specifica previsione che vuole essere effettuata.
    \end{itemize}
    Le stazioni appaltanti e gli enti concedenti hanno lo scopo di occuparsi del loro sviluppo, e, certamente uno degli elementi positivi che emerge è quello di una maggiore celerità dei processi, grazie a un'aumento dell'automazione che deve essere pur sempre dominata da parte della figura umana. Nel caso debbano affrontare delle problematiche, prima dell'indizione della gara, devono essere stabilite le clausole per poter gestire i possibili disguidi.
    Tutte queste soluzioni tecnologiche che abbiamo appena visto aumentano formalmente l'efficienza, formalmente poiché bisogna valutarne l'utilizzo concreto, così da riuscire a darne un giudizio oggettivo.\\
\end{justify}

\newpage\subsection{AI ACT e PNRR}
\begin{center}
    \includegraphics[width=0.3\linewidth]{Numero investimenti.png}
\end{center}
\begin{center}
    Questo grafico rappresenta il numero degli investimenti effettuati per Stato nel settore AI e Data.\citep{OECD_AI}
\end{center}

\begin{center}
    \includegraphics[width=0.3\linewidth]{Somma degli investimenti.png}
\end{center}
\begin{justify}
    Questo invece lo rappresenta sulla base dei milioni di dollari spesi in questo settore.\citep{OECD_AI2}\\
    Come si può notare in termini di numero di investimenti l'Italia non è messa benissimo rispetto agli altri 3 Stati inseriti nel grafico, allo stesso tempo però notiamo che, a riguardo della somma degli investimenti in previsione del 2023, che non sono ancora state confermate con i dati reali, fanno capire che ci sarà un calo drastico un po' in tutti gli Stati: forse perché molte di queste tecnologie vengono ‘‘monopolizzate’’ da grandi Stati (USA Cina).
    Si consigli la visione del sito \url{https://oecd.ai/en/data?selectedArea=investments-in-ai-and-data} per visualizzare bene tutti i grafici e capirne meglio le differenze.\\
    In data 6 Dicembre 2023 si è finalmente riusciti a raggiungere l'approvazione sul testo generale che comporrà l'AI ACT da parte dell'UE, un accordo di per sè storico che porta l'Europa ad essere il primo continente a creare una vera e propria disciplina.\\
    Non è però tutto così facile, perché come affermato dalla Rai, serviranno circa 2 anni per far si che tutto possa riuscire ad entrare effettivamente in funzione come d'accordo stabilito.\citep{PA_Digitale_AI}\\
    Non abbiamo ancora un testo effettivo, che si presume verrà pubblicato in primavera 2024, sappiamo però che l'approccio centrale che si deduce dalle informazioni trapelate, è l'approccio risk based, che sarà suddiviso in:
    \begin{itemize}
        \item \texttt{rischio minimo}: si riferisce agli algoritmi utilizzati nelle piattaforme di streaming e musicali, per la visualizzazione delle personali preferenze, quindi con intervento successivo solo se necessario.
        \item \texttt{rischio alto}: quelli che prima ancora di essere commercializzati devono essere regolamentati.
        \item \texttt{rischio inaccettabile}: alterazione di identità e diritti fondamentali (dovrà essere inserita una valutazione d'impatto sui i diritti fondamentali).
        \item \texttt{rischio specifico per la trasparenza}: questo prevede due livelli di adozione verso le AI generative:
        \begin{enumerate}
            \item \texttt{LIVELLO}: obbligo di pubblicazione dei contenuti che vengono utilizzati per \\l'addestramento delle AI, questo al fine di riuscire a garantire un più corretto rispetto del copyright.
            \item \texttt{LIVELLO}: verso i sistemi con una maggiore potenza di calcolo è prevista la creazione di un documento per la valutazione dei rischi sistemici e la comunicazione verso l'AI office istituito presso l'UE: chi non rispetta ciò è sanzionabile dal 1,5\% al 7\% del fatturato globale.\citep{La_repubblica} 
        \end{enumerate}
    \end{itemize}
    All'interno della PA l'impatto maggiore si avrà nei sistemi ad alto rischio:‘‘\textit{infrastrutture critiche, formazione scolastica e professionale, servizi pubblici e privati essenziali, amministrazione della giustizia e processi democratici’’}. Per cercare di ridurre le tempistiche di implementazione (intorno ai 2 anni), si è istituito l'AI PACT, con la funzione specifica di invitare qualsiasi start-up, azienda, a dare una mano a riguardo.\citep{marchetti_ai_2023}\\
    Oltre al fatto che i riconoscimenti biometrici potranno essere utilizzati senza approvazione preventiva del giudice, per esigenze di polizia quali: reati gravi ex-post e terrorismo in tempo reale (in teoria questa parte sarebbe da adottare entro 12 mesi).\\
    Il 2023 della PA digitale italiana è stato un anno di transizione, si è passati dal periodo pandemico al post PNRR, tra le innovazioni si annoverano:
    \begin{itemize}
        \item Circa il 80\% delle PA ha un responsabile per la transizione digitale e circa 13000 PA hanno pubblicato il PIAO e 4 linee guida aggiornate del CAD.
        \item Si è raggiunto un totale di 38 mln di SPID e 40 mln di CIE attivi, con 16.405 enti che usufruiscono di PAGOPA.
        \item Hanno aderito alla piattaforma digitale nazionale dati 4774 enti e 1561 erogano i servizi.
        \item Per la sperimentale piattaforma notifiche SEND si hanno 1400 enti aderenti di cui 400 attivi.
        \item Pubblicati 8 avvisi PNRR con 740 mln messi a disposizione.\citep{AttuazionePNRR}
    \end{itemize}
\end{justify}

\centering
\section{Conclusioni e opinioni finali}
\begin{justify}
   Emerge quindi che ciò di cui oggi si ha più bisogno è conoscenza e dibattito pubblico sull'intelligenza artificiale. Tante sono le domande che potrebbero sorgere, ma ci si concentra su quelle che fino a questo momento sono sorte:
   \begin{enumerate}
       \item \texttt{Esiste un confine netto tra umano e artificiale?}\\
       Se si fosse seguita la pista senza una regolamentazione nel settore, è molto probabile che il confine tra i due sarebbe stato impercettibile, di conseguenza si sarebbe raggiunto un punto di commistione tra i due.
       \item \texttt{L’intelligenza artificiale ci ruberà davvero il lavoro?}\\
       Essendo che è stata introdotta una regolamentazione, almeno in Europa, è un segno di come ci sia una forte volontà di dominare queste tecnologie, affinchè rimangano e abbiano sempre un ruolo di sostegno delle attività compiute dall'essere umano.
       \item \texttt{Che impatto ambientale produce lo sviluppo e l'impiego di sistemi di IA?}\\
       ChatGPT, che attualmente è la più rilevante AI generativa, è in grado di elaborare dati ma anche di generarne di nuovi: quindi il potenziale è grandissimo. Però si parla meno del costo energetico di questi modelli, di fatto, il consumo è enorme: circa 175 miliardi di parametri vengono utilizzati da ChatGPT, con un consumo di 1287 megawatt e 522 tonnellate di CO2. Per cercare di ovviare a questo problema, dalla versione 3 in poi OpenAI ha smesso di darli ulteriori informazioni di apprendimento, ma, per la versione 5 si stimano 100 volte tanto questi consumi ed emissioni.
       Secondo una stima effettuata da parte dell'Agenzia internazionale dell'energia, la generazione di un’immagine su ChatGPT equivale a una ricarica da 0 a 100 di un recente smartphone (di fatto questo è uno dei temi toccati dall’AI ACT, non è quello esclusivo ma uno dei tanti).
       Infine, la retorica della energia green è un po’ una giustificazione fasulla, perché se si volessero alimentare tutti questi impianti, servirebbero talmente tanti pannelli solari che in Europa non si potrebbe più utilizzare energia elettrica.\citep{nast_se_2023}
   \end{enumerate}
   In conclusione, si può affermare che questo sarà un periodo dove verranno fatte tante speculazioni, e, fino a quando le discipline non entreranno effettivamente in vigore, non si potrà dare una risposta univoca a tutti questi quesiti.
\end{justify}

\newpage\begin{justify}
    \bibliography{Bibliografia}
    \bibliographystyle{plainnat}
\end{justify}

\end{document}