\documentclass{article}
\usepackage{graphicx}
\usepackage{amsmath}
\usepackage{cite}
\usepackage{color}
\usepackage{enumitem}
\usepackage{hyperref}
\usepackage{natbib}
\usepackage{tabularx}
\usepackage{natbib}

\title{\textbf{"RETI NEURALI E CHAT GPT"}}
\author{Alessandro Meloni GEPID}
\date{}
\begin{document}
\maketitle
\begin{abstract}
    Questo sarà una piccolo riassunto del contenuto del paper.
\end{abstract}

\centering \tableofcontents
\centering \newpage
\section{Le Reti neurali}

\flushleft \subsection{Cosa sono?}

\flushleft\subsection{Quando sono state istituite?}

\centering \newpage
\section{Il funzionamento}

\flushleft \subsection{Cosa fanno e a cosa servono?}

\flushleft\subsection{ChatGPT}

\flushleft \subsection{Piccolo progetto di rete neurale}

\centering \newpage
\section{Considerazioni finali}

\end{document}